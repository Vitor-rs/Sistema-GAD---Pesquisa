\chapter{Metodologia}  \label{cap:04}

A metodologia consiste num conjunto de etapas ordenadamente dispostas a serem executadas e que tenham por finalidade a investigação de fenômenos para a obtenção de conhecimentos. Basicamente, compõe-se de etapas dispostas de forma sistemática, obedecendo a uma forma sequencial. 

Sendo assim, para a elaboração de um Trabalho de Pesquisa em Tecnologia da Informação, é preciso responder detalhadamente as seguintes questões:\\

\textbf{Como se procederá a pesquisa?}

% enumerate é um comando que insere listas numeradas
\begin{enumerate}
    \item Qual será o tema da sua pesquisa? 
    \item Qual o espaço (local ou área) delimitado da pesquisa? 
    \item Qual é o pretende resolver?
    \item Qual será o tipo da sua pesquisa? Desenvolvimento de Software ou Pesquisa Bibliográfica?
    \item Se for realizar pesquisa Bibliográfica, qual será sua área de estudo? Que autores pretende abordar?
    \item Pretende realizar questionários ou entrevistas com pessoas da área?
    \item Se a pesquisa for de desenvolvimento de software, como será feita a análise de requisitos? Quem será consultado? 
    \item Como será construída a documentação do Sistema?
    \item Quais serão as tecnologias utilizadas para a documentação, desenvolvimento e testes de software?
    \item Como o sistema será construído?
    \item Como será implementado o sistema?
    \item Se for realizar testes de software, qual método pretende adotar?
\end{enumerate}
