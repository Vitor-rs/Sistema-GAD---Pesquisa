\chapter{Introdução}

A introdução é um capítulo tão importante quanto qualquer outro. Nele você deve apresentar o tema do seu trabalho, bem como a delimitação que será assumida e a problemática que será analisada por você. A introdução do TCC deve explicar o contexto em que o tema está inserido e deixar claro o motivo do assunto ser importante e relevante para sua área de pesquisa.

Dessa maneira, para escrever uma introdução que aborde esses aspectos, podemos seguir a seguinte estrutura:

% Esse ambiente Insere uma lista não Numerada
\begin{itemize}
    \item \textbf{Apresentação do Tema e do Contexto da área da Pesquisa:} Devemos pensar que o leitor não conhece nosso tema e área da pesquisa na qual estamos atuando, devemos então explicar brevemente para ele do que se trata nosso trabalho e a área na qual estamos estudando. Por exemplo, se nosso estudo propõe um aplicativo de celular para a educação, podemos começar a introdução falando do uso dos aplicativos de Smarthphone para aproveitamento de conteúdo.  
    
    \item \textbf{Delimitação do Tema:} Nesse momento, após o leitor conhecer o tema de maneira ampla, podemos delimitar o assunto somente a área de estudos que iremos assumir. Por exemplo, se fora dito que o trabalho irá atuar na área de desenvolvimento de aplicativos para a educação, nesse momento iremos mostrar que tipo de desenvolvimento será feito, quais ferramentas e ambientes serão utilizados e qual área ele atende, no caso, sendo educação especial ou uma matéria de estudo.
    
    \item \textbf{Problemática:} Nesse parágrafo você deve mostrar ao seu leitor qual é o problema que você pretende solucionar. Sendo uma hipótese de um dado assunto ou o desenvolvimento de um software, é aqui que você deve mostrar porque sua ideia é diferente de todas as outras para responder uma problemática em comum. Por exemplo, ao iniciar o trabalho, se minha hipótese é "Um aplicativo de celular ajuda os alunos a aprender melhor um conteúdo de matemática?". É nesse paragrafo que eu devo mostrar ao meu leitor como surgiu essa pergunta e quais são as possíveis respostas que eu pretendo descobrir.
    
    \item \textbf{Objetivos e Metodologia}: Por fim, você deve explanar rapidamente sobre os objetivos que deseja alcançar com todo o seu projeto e a Metodologia Proposta.
\end{itemize}

Se eu posso dar uma dica, fica mais confortável escrever a Introdução depois de ter desenvolvido os Objetivos e a Metodologia do Trabalho.
E lembre-se, só escreve bem sobre um assunto quem lê e estuda muito ele.


