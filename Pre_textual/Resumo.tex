\begin{resumo}

\textbf{Resumo:} Sistemas acadêmicos brasileiros, como SIGAA, SIGA, SUAP e similares, oferecem registro e validação digital de atividades complementares, mas ainda dependem da análise manual dos certificados por docentes avaliadores, conforme evidenciado nas práticas de UFSC, UFC e UDESC. O processo envolve etapas burocráticas, exigência de documentos oficiais, assinaturas digitais e categorização segundo regras específicas de cada instituição, tornando-o suscetível a erros e pouco escalável. Este trabalho propõe o Sistema GAD como um módulo especializado, integrável via API ou componente local, capaz de processar certificados e auto-categorizá-los com base em heurísticas e regras acadêmicas institucionais. Utilizando extração de texto em tempo real e aprendizado de máquina, o GAD visa automatizar o preenchimento dos formulários, reduzir o esforço humano, aumentar a eficiência e garantir conformidade, rastreabilidade e segurança na gestão das atividades acadêmicas.

\vspace{\onelineskip}

\textbf{Palavras-chave}: automação, certificados acadêmicos, integração modular, sistemas brasileiros, aprendizado de máquina.

\end{resumo}