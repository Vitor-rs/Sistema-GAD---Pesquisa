\begin{resumo}

%\noindent{SILVA, João da. \textbf{Título do trabalho de conclusão de curso em negrito}. Nº de fls. TCC (Trabalho de Conclusão de Curso). Instituto Federal de Mato Grosso do Sul – IFMS. Tecnologia em Análise e Desenvolvimento de Sistemas, Câmpus Nova Andradina, MS. 2018.}

%\setlength{\absparsep}{18pt} % ajusta o espaçamento dos parágrafos do resumo
%\vspace{1.5cm}
	
\textbf{Resumo:} O resumo é um pequeno texto sobre o trabalho que ressalta informações importantes sobre ele. Cabe ao estudante destacar informações seguindo um padrão de escrita para maior clareza e completude de seu texto. \textbf{Contexto:} Neste trecho é explicado o contexto geral de aplicação do tema escolhido. \textbf{Objetivo:} Aqui o estudante informa os objetivos gerais do trabalho em texto corrido. \textbf{Métodos:} Descrever as metodologias utilizadas no trabalho. \textbf{Resultados:} Descrever os principais resultados do trabalho de modo sucinto \textbf{Conclusão:} Apresentar conclusões sucintas do trabalho.

	\vspace{\onelineskip}
	
	\textbf{Palavras-chave}: Inserir no máximo 5 palavras chaves separadas por vírgula.
	
\end{resumo}