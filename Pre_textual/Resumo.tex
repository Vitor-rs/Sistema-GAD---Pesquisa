\begin{resumo}


\textbf{Resumo:} Sistemas acadêmicos brasileiros, como SIGAA, SIGA, SUAP e similares, oferecem funcionalidades para registro, validação digital e creditação de atividades complementares, porém ainda dependem fortemente da análise manual dos certificados por docentes avaliadores. Essa abordagem, presente em instituições como UFSC, UFC e UDESC, resulta em processos burocráticos, suscetíveis a erros e pouco escaláveis diante do grande volume de documentos submetidos. O presente trabalho propõe o desenvolvimento do Sistema GAD, uma solução web inteligente capaz de automatizar o preenchimento e categorização de formulários de certificados estudantis, utilizando extração de texto em tempo real e técnicas de aprendizado de máquina. O objetivo é reduzir o esforço humano, aumentar a eficiência institucional e garantir maior padronização, rastreabilidade e segurança na gestão das atividades acadêmicas.

\vspace{\onelineskip}

\textbf{Palavras-chave}: automação, certificados acadêmicos, sistemas brasileiros, aprendizado de máquina, extração de texto.

\end{resumo}