\begin{resumo}

\textbf{Resumo:} A validação de atividades complementares nos cursos superiores brasileiros, como ADS/IFMS, UFSC, UFC e UDESC, é marcada por exigências normativas rigorosas, categorização detalhada (ensino, pesquisa, extensão, inovação) e processos burocráticos que demandam análise manual de certificados por docentes. Cada instituição adota sistemas digitais (SIGAA, SIGA, SUAP), mas a automação é limitada: o estudante precisa reunir documentos oficiais, preencher formulários e aguardar deferimento, com regras específicas de carga horária, percentuais máximos por modalidade e critérios de autenticidade. O processo é suscetível a erros, retrabalho e baixa escalabilidade, dificultando o acompanhamento e a integralização curricular. O Sistema GAD propõe uma solução automatizada, integrável via API, capaz de extrair texto de certificados, categorizar atividades conforme regulamentos institucionais e preencher formulários de forma inteligente, reduzindo o esforço humano, aumentando a rastreabilidade e garantindo conformidade legal. Essa abordagem visa transformar a gestão das atividades complementares, promovendo eficiência, transparência e segurança acadêmica.

\vspace{\onelineskip}

\textbf{Palavras-chave}: atividades complementares, automação, validação acadêmica, sistemas digitais, conformidade institucional.

\end{resumo}