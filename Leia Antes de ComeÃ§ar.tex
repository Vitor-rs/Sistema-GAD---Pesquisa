Oi.
Você acaba de baixar o Template de TCC do TADS do IFMS.
Para tal, deve seguir algumas regras importantes.

Para começar, você irá ver muitos códigos inseridos nos arquivos. São eles que garantem que seu texto seja formatado de acordo com as regras ABNT vigentes. Não os apague ou edite sem saber realmente o que está fazendo.

Sendo assim, você pode observar ao lado a árvore de arquivos. Nessa árvore estão dispostas as estruturas que você irá precisar no decorrer da produção do TCC. Você pode observar a disposição dos arquivos em pastas por assunto:

Raiz: 
- edicoes.sty: Esse é o arquivo que você irá fazer as modificações propriamente ditas para capa e contra-capa. Nele você irá encontrar as instruções de edição desse documento.
- Monografia\_config.tex: Nesse arquivo foram programados os comandos necessários para a formatação desse arquivo. Cada seção está comentada para que você saiba o que significa, porém, evite edições. Prevendo que talvez você precise inserir novos capítulos, ao final está marcado e instruído como você deve chamar novos arquivos.  
- referencias.bib: Nessa seção você irá inserir a bibliografia utilizada para referenciamento das citações. No tutorial oferecido há um capítulo inteiro sobre como você deve montar referências de acordo com o uso. Consulte-o.

Pasta Capítulos: (Elementos Textuais)
- Divididos em Introdução, Objetivos, Referencial e tudo mais.
- Cada um desses Capítulos contém um texto de auxílio para a produção do TCC. Leia-o e apague-o para começar sua edição. Caso deseje preservá-lo, faça uma cópia do projeto antes de apagar e deixe-o apenas para consulta desses textos;
- Evite renomear os arquivos já criados, pois já foram chamados para a impressão;
- Você pode criar novos capítulos desde que se lembre de chamá-los no Arquivo de Configuração /monografia\_config.

Ficha Catalográfica e Folha de Aprovação: 
- Esses elementos estão inseridos no arquivo e configurados, mas como eles somente serão usados na última versão do TCC, pós-defesa de banca, foram comentados para não aparecer na compilação do texto. Quando você necessitar da utilização, basta ir no arquivo de configuração e retirar o símbolo de comentário para que as páginas apareçam no arquivo final.
- A folha de aprovação deve ser fornecida pela Banca após a apresentação final, devidamente assinada.
- A Ficha catalográfica é fornecida pela biblioteca da Instituição.

Pasta Imagens:
- Nela você deve fazer upload de imagens que irá usar no TCC. Lembre-se de upar apenas os arquivos que irá usar, pois, como fica hospedado na nuvem, ficam pesados.
- Faça upload de arquivos com extensões mais leves para que não pese na compilação do projeto.
- O modo de como upar imagens e inserir no texto está descrito no tutorial fornecido.

Pasta Pré Textual:
- Todas as estruturas necessárias para a produção do TCC foram inseridas nessa pasta e chamadas no arquivo de Configuração. Porém, nem todas serão usadas nas primeiras versões.
- Para chamar esses arquivos no texto, basta procurar suas chamadas no arquivo de configuração que estão comentadas, e retirar o símbolo do comentário.
- Documentos obrigatórios na Primeira versão: Capa, Folha de Rosto, Resumo.
- Documentos Obrigatórios na Segunda versão: Capa, Folha de Rosto, Resumo, Abstract, (Opcionais: Agradecimento, dedicatória e epígrafe).
- Documentos Obrigatórios na  versão final: Capa, Folha de Rosto, Resumo, Abstract, Ficha catalográfica, Folha de Aprovação,  (Opcionais: Agradecimento, dedicatória e epígrafe). Você pode ainda inserir a lista de Figuras, que já está pré configurada, basta retirar o comentário do arquivo de configuração.

Leia sempre as indicações dos comentários presentes nos arquivos. Evite apagá-los para que possa se lembrar deles.

Enfim, qualquer dúvida que você possa ter, consulte o Tutorial disponibilizado. Caso não encontre a resposta para sua dúvida, pesquise nos forúns de Latex na Internet.








